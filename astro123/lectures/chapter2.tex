\chapter{Jak pulsují hvězdy}

\section{Numerické řešení obyčejných diferenciálních rovnic}

S nutností řešit obyčejné diferenciální rovnice se můžeme setkat na každém kroku již během studií fyziky. Když pomineme
učebnicové případy, většinu z nich nelze řešit analyticky a musíme se uchýlit k numerickým metodám. V následující kapitole si ukážeme nejběžněji používané metody pro jejich řešení, které jsou součástí mnoha numerických knihoven v různých programovacích jazycích, včetně jazyka PYTHON. 

Nejprve se zaměříme na obyčejné diferenciální rovnici I. řádu

\begin{equation}
\label{math:ode_1order}
\deriv{y(x)}{x} = f(x,y),
\end{equation}

závěry a zkušenosti pak zobecníme na soustavu $N$ obyčejných diferenciálních rovnic I. řádu

\begin{equation}
\label{math:ode_set_1order}
\deriv{\vec{y}}{x} = \vec{f}(x,\vec{y}),
\end{equation}

kde $\vec{y}(x) = (y_1(x),y_2(x),\dots,y_n(x))$. Je třeba si uvědomit, že se nejdá o žádné omezení. Diferenciální rovnice vyšších řádů je vždy možné nahradit soustavou rovnic 1.řádu. Dále se omezíme na řešení počátečních úlohy ({\it Cauchyho úlohy}), kde k diferenciálním rovnicím máme ještě specifikované počáteční podmínky, přičemž hledáme řešení této rovnice na nějakém intervalu $x \in <a,b>$. Počáteční podmínky v případě 
(\ref{math:ode_1order}) jsou dány $y(x_0) = y_0$ , nebo v případě soustavy rovnic (\ref{math:ode_set_1order}) $\vec{y}_0 = (y_1(x_0),y_2(x_0),\dots,y_n(x_0))$.

Základem řešení rovnice  (\ref{math:ode_1order}) je její diskretizace. Spočívá v nahrazení spojitého průběhu funkce, jejich derivací i nezávislé proměnné veličiny sadou diskrétních bodů. Vytvoříme na intervalu $<a,b>$ síť ekvidistantních bodů, kterou aproximuje spojitou nezávislou veličinu $x$

\begin{equation}
a = x_0 < x_1 < x_2 < \dots < x_N = b
\end{equation}

Předpoklad ekvidistantních bodů není nutný, nicméně je pro běžné úlohy velmi častý a celou situaci zjednodušuje.

\subsection{Jednokrokové metody}
 S pomocí této sítě aproximujeme potřebnou derivaci a stačí nám k tomu Taylorův rozvoj. Taylorův rozvoj funkce $y$ v bodě $x+h$

\begin{equation}
y(x+h) = y(x)+{\deriv{y}{x}}h+{\frac{\rm d^2}y}{{\rm d}x^2}h^2+\dots
\end{equation}

kde $h$ je malý increment. Pokud zanedbáme členy druhého řádu a vyšší můžeme obratem vyjádřit aproximaci derivace s využitím sítě ekvidistantních bodů $x_k$

\begin{equation}
\deriv{y}{x} = \frac{y_{k+1}-y_{k}}{x_{k+1}-x_{k}}+O(h^) \quad k = 0,\dots, N-1,
\end{equation}

$O(h^2)$ značí chybu druhého řádu, index $k$ značí hodnotu funkce $y$ odpovídající diskrétní hodnotě proměnné $x_k$. Dosazením do rovnice (\ref{math:ode_1order}) a drobnou úpravou dostáváme

\begin{equation}
y_{k+1} = y_{k}+f(x_k,y_k)h_k \quad {\text{kde}} \quad x_{k+1}=x_k+h
\end{equation}

Tato jednoduchá metoda numerického řešení ODE se nazýva {\it Eulerova} metoda. Vzhledem k tomu, že jsme v během odvozování v Taylorově rozvoje zanedbali členy druhého řádu, jedná se o méně přesnou metodu prvního řádu. Metoda sama patří mezi {\it jednokrokové metody}, protože k závisí na informaci jednoho bodu. Přesněji řečeno, k tomu abychom určily další bod funkce $y_{k+1}$ stačí nám k tomu informace z bodu $y_k$.

\subsection{Vícekrokové metody}
Nevýhodou předchozí zmíněné metody pro řešení ODE je její přesnost. Abychom docílili vyšší přesnosti museli bychom v Taylorově rozvoji použít více členů
\begin{equation}
y(x+h) = y(x)+\deriv{y}{x}h+\frac{1}{2}\frac{{\rm d}^2 y}{{\rm d} x^2}h^2+\frac{1}{6}\frac{{\rm d}^3 y}{{\rm d} x^3}h^3+\dots
\end{equation}

Tím však vyvstává problém, protože v diskretní reprezentaci pro metodu druhého řádu vyvstává nutnost určení derivace vyšších řádů. Konkrétně v našem případě druhého řádu

\begin{equation}
\label{math:taylor_2order_runge}
y_{k+1} = y_{k} + \left.\deriv{y}{x}\right|_{k}
          h_k + \frac{1}{2} \left.\frac{{\rm d}^2 y}{{\rm d}x^2}\right|_{k} h_{k}^2
\end{equation}

S použitím pravidla pro derivaci složené funkce, můžeme pro diferenciaci $\deriv{y}{x}=f(x,y)$ psát

\begin{equation}
\frac{{\rm d}^2 y}{{\rm d}x^2} = f_x(x,y)+f_y(x,y) \deriv{y}{x} = f_x(x,y)+f_y(x,y) f(x,y)
\end{equation}

následně využijeme Taylorův rozvoj pro funkci dvou proměnných, s
pomocí které se budeme snažit vyjádřit druhou derivaci  

\begin{equation}
f(x+h,y+hf) = f(x,y)+h \frac{{\rm d}f}{{\rm d}x}+h\frac{{\rm d}f}{{\rm d}y}f+O(h^2)
\end{equation}

skombinováním obou rovnic dostáváme rovnici

\begin{equation}
\frac{{\rm d}f}{{\rm d}x}+h\frac{{\rm d}f}{{\rm d}y}f =
\frac{f(x+h,y+hf)-f(x,y)}{h}+O(h^2)
\end{equation}

Aproximaci druhé derivace s použitím diskrétní reprezentace využijeme
pro úpravu (\ref{math:taylor_2order_runge})

\begin{eqnarray}
y_{k+1} = y_k+f(x_k,y_k)h_k+\frac{f(t_k+h_k,y_k+h_k
  f(x_k,y_k))}{2h_k}h_k^2= \\
y_k+\frac{f(x_k,y_k)+f(x_k+h_k,y_k+h_k f(x_k,y_k))}{2}h_k
\end{eqnarray}

Už se blížíme k finálnímu výrazu, stačí jenom použít substituci

\begin{eqnarray}
k_1 = f(x_k,y_k)h_k \\
k_2 = f(x_k+h_k,y_k+k_1)h_k
\end{eqnarray}


\section{Jak pulsují hvězdy}
V reálné praxi astrofyzika se nutností řešení diferenciláních rovnic
setkává prakticky na každém kroku. Problémy nebeské mechaniky,
určování pohybu těles ve sluneční soustavě, teorii hvězdných atmosfér,
hvězdné stavby a mnohé další. My si jako ilustrační příklad vezmeme
nám blízký problém hvězdných pulsací. Z kurzu {\it Základů astronomie}
si určitě vzpomínáte, že některé hvězdy nezůstávají ve hydrostatické
rovnováze, ale namísto toho hvězdná obálka rozpíná a smršťuje - pulzuje.

Uvedem si teď jednoduchý model pulsující hvězdy - cefeidy, který však vede k poměrně 
přesným závěrům , zejména ve vztahu k periodě. Model se skládá z centrálního bodu reprezentující celou hvězdu o hmotnosti 
$M$, která je obklopena tenkou sféricky symetrickou obálkou o hmotnosti $m$ a poloměru $R$, která reprezentuje povrch hvězdy.
Vnitřek hvězdy je vyplněn plynem se zanedbatelnou hmotností a tlakem, který kompenzuje přitažlivý gravitační vliv centrální hvězdy na obálku. S použitím Newtonových rovnic můžeme psát

\begin{equation}
m\frac{{\rm d}^2 R}{{\rm d} t^2} = -\frac{GMm}{R^2}+4\pi R^2 P.
\end{equation}

kterou dále přepíšeme na soustavu

\begin{eqnarray}
\label{math:oscil_set_pressure}
\deriv{R}{t} = v \\
\deriv{v}{t} = -\frac{GM}{R^2} + 4\pi R^2 P/m.
\end{eqnarray}

Ještě se musíme vypořádat s tlakem, poslouží nám stavová rovnice, kdy budeme předpokládat, že oscilace od rovnovážné polohy dané poloměrem $R_0$ jsou adiabatické, tedy platí vztah

\begin{equation}
PV^{\gamma} = konst. \rightarrow {\quad} P_0 V_0^{\gamma} = P V^{\gamma}
\end{equation}

S použitím vztahu pro objem plynu v kouli $V = \frac{4}{3}\pi R^3$, můžeme dále psát

\begin{equation}
P_0 R_0^3 = P R^3 \rightarrow {\quad} P = P_0 \left(\frac{R_0}{R}\right)^{3\gamma}
\end{equation}

Dosazením do soustavy (\ref{math:oscil_set_pressure}) dostáváme finální podobu soustavu obyčejných diferenciálních rovnic popisujícich hradiální hvězdné pulsace

\begin{equation}
\label{math:oscil_final_set}
\deriv{R}{r} = v \\
\deriv{v}{t} = -\frac{GM}{R^2}+4\pi R^2 \frac{P_0}{m} \left(\frac{R_0}{R}\right)^{3\gamma}
\end{equation}

Tuto soustavu budeme řešit numericky, je však ještě třeba ji doplnit hodnoty parametrů a počáteční podmínky. Naší modelovou hvězdou bude hvězda 
$\delta$ Cep
